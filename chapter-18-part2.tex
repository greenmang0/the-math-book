\documentclass{article}
\usepackage{amsmath}
\usepackage{amssymb}
\usepackage{amsfonts}
\usepackage{paracol}
\usepackage[inline]{enumitem}
\usepackage{geometry}
\renewcommand{\arraystretch}{1.2}
\geometry{a4paper, left=15mm, top=15mm, right=15mm, bottom=15mm}
\title{Chapter 18

  Advanced Trigonometry}
\date{}
\begin{document}

\paragraph{Q3.}
In $\triangle ABC, \angle A = 83^{\circ}, \angle B= 50^{\circ}$ and $CB=23.8$, find $AC(b)$ and $AB(c)$

{\scriptsize \textbf{Solution:}}

\[
  \begin{aligned}
    \frac{23.8}{\sin 83^{\circ}} = \frac{b}{\sin 50^{\circ}} \\
    \therefore b = 18.37 \\
    \\
    \\
    \angle C = 180^{\circ}-50^{\circ}-83^{\circ}=47^{\circ} \\
    \frac{23.8}{\sin 83^{\circ}} = \frac{c}{\sin 47^{\circ}} \\
    \therefore c = 17.54
  \end{aligned}
\]

\paragraph{Q4.}
Find $BC$ and $CD$

{\scriptsize \textbf{Solution:}}

\[
  \begin{aligned}
    \frac{a}{\sin 15^{\circ}} = \frac{10}{\sin 12^{\circ}} \\
    \therefore a = 12.45 = BC \\
    \\
    \\
    \angle BCD = 12^{\circ}+15^{\circ}=27^{\circ} \\
    \cos 27^{\circ} = \frac{CD}{BC} = \frac{CD}{12.45} \\
    CD = 12.45 \cos 27^{\circ} \\
    \therefore CD = 11.09 \text{ units}
  \end{aligned}
\]

\paragraph{Q5.}
Find $\cos \angle WDY$

{\scriptsize \textbf{Solution:}}

\[
  \begin{aligned}
    WY^{2} = WX^{2} + XY^{2} \\
    WY = \sqrt{8^{2}+6^{2}} = 10 \\
    DY = \sqrt{DC^{2}+CY^{2}} = \sqrt{6^{2}+5^{2}} = 7.81 \\
    WD = \sqrt{WZ^{2}+DZ^{2}} = \sqrt{8^{2}+5^{2}} = 9.43 \\
    \cos \angle WDY = \frac{9.43^{2}+7.81^{2}-10^{2}}{2\times9.43\times7.81} = 0.34 \\
    \therefore \cos \angle WDY = 70.2^{\circ}
  \end{aligned}
\]

\paragraph{Q6.}
If the three sides $AB, BC$ and $AC$ of a $\triangle$ are $2x, 2x+3$ and $5$, $\angle ABC=60^{\circ}$, find $x$.

{\scriptsize \textbf{Solution:}}

Applying cosine rule
\[
  \begin{aligned}
    \cos 60^{\circ} = \frac{2x^{2}+(2x+3)^{2}-5^{2}}{2(2x)(2x+3)} \\
    \frac{1}{2} = \frac{4x^{2}+4x^{2}+12x+9-25}{8x^{2}+12x} \\
    16x^{2}+24x-32 = 8x^{2}+12x \\
    8x^{2}+12x-32 = 0 \\
    4(2x^{2}+3x-8) = 0 \\
    2x^{2}+3x-8 = 0 \\
    x = \frac{-3 \pm \sqrt{9 - 4 \times 2x(-8)}}{2\times2} \\
    x = \frac{-3 \pm \sqrt{73}}{4} \\
    \therefore x = 1.39, -2.89
  \end{aligned}
\]

\paragraph{Q7.}
In figure, $P(2, 4), Q(4, 8)$ and $R(4, 0)$ are three vertices of a $\triangle PQR$. Find
\begin{enumerate*}[label=\roman*)]
  \item $PQ$
  \item $PR$
  \item $QR$
  \item $\angle PQR$
  \item $\angle QPR$
\end{enumerate*}

{\scriptsize \textbf{Solution:}}

\[
  \begin{aligned}
    PQ = \sqrt{(x_{2}-x_{1})^{2}+(y_{2}-y_{1})^{2}} \\
    \therefore PQ = \sqrt{(4_{2}-2_{1})^{2}+(8_{2}-4_{1})^{2}} = 2\sqrt{5} \\
    \\
    \\
    PR = \sqrt{(4_{2}-2_{1})^{2}+(0_{2}-4_{1})^{2}} = 2\sqrt{5} \\
    \therefore \triangle \text{PQR is isosceles.} \\
    \\
    \\
    QR = \sqrt{(4_{2}-4_{1})^{2}+(0_{2}-8_{1})^{2}} = \sqrt{0+64} = 8 \\
    \\
    \\
    \text{Applying cosine rule} \\
    \cos \angle PQR = \frac{(\sqrt{20})^{2}+8^{2}-(\sqrt{20}^{2})}{2\sqrt{20}\times 8} \\
    \cos \angle PQR = 0.8944 \\
    \cos \angle PQR = 26.6^{\circ} \\
    \text{Now, } PQ = PR \\
    \therefore \angle PQR = \angle PRQ = 26.6^{\circ} \\
    \\
    \\
    \angle QPR = 180^{\circ}-(26.6^{\circ}+26.6^{\circ}) = 126.8^{\circ}
  \end{aligned}
\]

\paragraph{Q8.}
Find $\theta$ in the adjoining figure, given $AB=8, BC=6, AD=4$.

{\scriptsize \textbf{Solution:}}
Applying Pythagoras theorem

\[
  \begin{aligned}
    AC = \sqrt{AB^{2}+BC^{2}} = \sqrt{8^{2}+6^{2}} = 10 \\
    \text{Applying cosine rule} \\
    \cos \theta = \frac{10.77^{2}+10^{2}-4^{2}}{2 \times 10.77 \times 10} \\
    \cos \theta = 0.93 \\
    \therefore \theta = 21.8^{\circ}
  \end{aligned}
\]

\paragraph{Q9.}
In the adjoining figure, $A=11, \angle C = 40^{\circ}, \angle B=95^{\circ}$.

{\scriptsize \textbf{Solution:}}

\[
  \begin{aligned}
    \frac{AC}{\sin \angle ABC} = \frac{AC}{\sin \angle ACB} \\
    \frac{11}{\sin 95^{\circ}} = \frac{AB}{\sin 40^{\circ}} \\
    \therefore AB = \frac{11 \sin 40^{\circ}}{\sin 95^{\circ}} = 7.10 \\
    \\
    \\
    \angle A = 180^{\circ}-95^{\circ}-40^{\circ} = 45^{\circ} \\
    \frac{AB}{\sin \angle ACB} = \frac{BC}{\sin 45^{\circ}} \\
    \frac{7.10}{\sin 40^{\circ}} = \frac{BC}{\sin 45^{\circ}} \\
    \therefore BD = \frac{7.10 \times \sin 45^{\circ}}{\sin 40^{\circ}} = 7.81
  \end{aligned}
\]

\paragraph{Q10.}
A pyramid of square base of $6m$ side has vertical height $9m$. $M$ is the midpoint of $AB$. Find
\begin{enumerate*}[label=\roman*)]
  \item $VA$
  \item $\angle$ of face $VAB$ makes with base of pyramid.
  \item Volume of pyramid
  \item Surface area of pyramid
\end{enumerate*}

{\scriptsize \textbf{Solution:}}

\[
  \begin{aligned}
    AC = \sqrt{AB^{2}+BC^{2}} = \sqrt{6^{2}+6^{2}} = 8.48m \\
    AX = \frac{1}{2}AC = 4.24 \\
    \therefore VA = \sqrt{VX^{2}+AX^{2}} = \sqrt{9^{2}+4.24^{2}} = 9.95m \\
    \\
    \\
    MX = \frac{1}{2}6 = 3m \\
    \tan \angle VMX = \frac{VX}{MX} = \frac{9}{3} = 3 \\
    \therefore \angle VMX = \tan^{-1}3 = 71.6^{\circ} \\
    \\
    \\
    \text{Volume} = \frac{1}{3} \times base \times height \\
    \therefore \text{Volume} = \frac{1}{3} \times 6 \times 6 \times 9 = 108m^{3}
    \\
    \\
    VM = \sqrt{VA^{2}-AM^{2}} = \sqrt{9.95^{2}-3^{2}} = 9.49m \\
    \text{Area of } \triangle ABV = \frac{1}{2}VM \times AB = \frac{1}{2} \times 9.49 \times 6 \\
    \text{Area of } \triangle ABV = 28.46m^{2} \\
    \text{Total surface area} = \text{Area of four } \triangle s + \text{Area of base} \\
    \therefore \text{Total surface area} = 4(28.46) + 6 \times 6 = 150m^{2}
  \end{aligned}
\]

\paragraph{Q11.}
In a square based pyramid, a) calculate $VC$ (slant edge). b) Calculate the angle $VC$ makes with $AC$.

{\scriptsize \textbf{Solution:}}

\[
  \begin{aligned}
    AC = \sqrt{AB^{2}+BC^{2}} = \sqrt{50^{2}+50^{2}} = \sqrt{200} = 70.7m \\
    \angle VXC = 90^{\circ}, XC = \frac{1}{2}(70.7) = 35.35m \\
    \therefore VC = \sqrt{VX^{2}+XC^{2}} = \sqrt{7^{2}+35.35^{2}} = 36.0m \\
    \\
    \\
    \sin \angle VCX = \frac{7}{36} \\
    \therefore \angle VCX = \sin^{-1} \left( \frac{7}{36} \right) = 11.2^{\circ}
  \end{aligned}
\]

\paragraph{Q12.}
In the rectangular box $ABCDPQRS$, given $AB=10, BC=5, BQ=7$, find (a) Largest Diagonal (b) $\angle QBD$

{\scriptsize \textbf{Solution:}}

\[
  \begin{aligned}
    DB = \sqrt{10^{2}+5^{2}} = \sqrt{125} \\
    \angle DBQ = 90^{\circ} \\
    DQ = \sqrt{DB^{2}+QB^{2}} = \sqrt{125+49} = 13.2 \\
    Longest diagonal = 13.2 units \\
    \\
    \\
    \tan \angle QDB = \frac{DB}{DQ} = \frac{\sqrt{125}}{13.2}
  \end{aligned}
\]

\paragraph{Q13.}
The cube shown as edges of length $20cm$. The midpoint of $AP$ is $M$. Calculate the length of (a) $AC$ (b) $CM$ (c) $\angle CMR$

{\scriptsize \textbf{Solution:}}

\[
  \begin{aligned}
    AC = \sqrt{AD^{2} + DC^{2}} = \sqrt{20^{2} + 20^{2}} = \sqrt{800} = 20 \sqrt{2} = 28.3cm \\
    \\
    \\
    CM = \sqrt{AM^{2} + AC^{2}} = \sqrt{20^{2} + 800} = 30cm \\
    \\
    \\
    \text{Let N be the midpoint of CR} \\
    \triangle MCR \text{ is an isosceles triangle} \\
    \therefore MN \text{ is perpendicular} \\
    \sin x = \frac{10}{30} \Rightarrow x = \sin^{-1}\frac{1}{3}  = 19.45^{\circ} \\
    \therefore 2x = 38.9^{\circ}
  \end{aligned}
\]


\end{document}
