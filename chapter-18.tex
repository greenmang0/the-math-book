\documentclass{article}
\usepackage{amsmath}
\usepackage{amssymb}
\usepackage{amsfonts}
\usepackage{paracol}
\usepackage[inline]{enumitem}
\usepackage{geometry}
\geometry{a4paper, left=15mm, top=15mm, right=15mm, bottom=15mm}
\title{Chapter 18

  Advanced Trigonometry}
\begin{document}
\maketitle
\section{Radian and Degree Measures}
% example 1
\subsection{Convert them into degree measure :}
\begin{enumerate}
        \begin{paracol}{3}
          \item[a.] $\frac{\pi}{6}$
          \item[d.] $\frac{7\pi}{6}$
          \item[g.] $\frac{7\pi}{18}$
          \switchcolumn
          \item[b.] $\frac{4\pi}{5}$
          \item[e.] $\frac{8\pi}{3}$
          \switchcolumn
          \item[c.] $\frac{11\pi}{6}$
          \item[f.] $\frac{7\pi}{4}$
          \switchcolumn
        \end{paracol}
\end{enumerate}
{\footnotesize Since $\pi=180^{0}$, $1 radian = \frac{180^{0}}{\pi}$ OR $1^{0} = \frac{\pi}{180^{0}}$}

\paragraph{Solutions :}
\begin{enumerate}
        \begin{paracol}{3}
          \item[a.] $\frac{\pi}{6} \times \frac{180^{0}}{\pi} = 30^{0}$
          \item[d.] $\frac{7\pi}{6} \times \frac{180^{0}}{\pi} = 210^{0}$
          \item[g.] $\frac{7\pi}{18} \times \frac{180^{0}}{\pi} = 70^{0}$
          \switchcolumn
          \item[b.] $\frac{4\pi}{5} \times \frac{180^{0}}{\pi} = 144^{0}$
          \item[e.] $\frac{8\pi}{3} \times \frac{180^{0}}{\pi} = 480^{0}$
          \switchcolumn
          \item[c.] $\frac{11\pi}{6} \times \frac{180^{0}}{\pi} = 330^{0}$
          \item[f.] $\frac{7\pi}{4} \times \frac{180^{0}}{\pi} = 315^{0}$
          \switchcolumn
        \end{paracol}
\end{enumerate}

% example 2
\subsection{Convert them to radian measure :}
\begin{enumerate}
        \begin{paracol}{3}
          \item[a.] $\theta=5^{'}$
          \item[d.] $330^{0}$
          \item[g.] $520^{0}$
          \switchcolumn
          \item[b.] $-120^{0}$
          \item[e.] $480^{0}$
          \item[h.] $240^{0}$
          \switchcolumn
          \item[c.] $210^{0}$
          \item[f.] $-47^{0}30^{'}$
          \item[i.] $-7^{0}30^{'}$
          \switchcolumn
        \end{paracol}
\end{enumerate}

\paragraph{Solutions :}
\begin{enumerate}
        \begin{paracol}{3}
          \item[a.] $\theta=5^{'}=\frac{5^{0}}{60} \times \frac{\pi}{180^{0}} = \frac{\pi}{12 \times 180} radians$
          \item[d.] $330^{0} = 330^{0} \times \frac{\pi}{180^{0}} = \left(\frac{11\pi}{6}\right)^{c}$
          \item[g.] $520^{0} = -520 \times \frac{\pi}{180^{0}} = \frac{26\pi}{9} $
          \switchcolumn
          \item[b.] $-120^{0} \times \frac{\pi}{180^{0}} = \left(\frac{-2 \pi}{3}\right)^{c}$
          \item[e.] $480^{0} = 480^{0} \times \frac{\pi}{180^{0}} = \frac{8\pi}{3}$
          \item[h.] $240^{0} = 240^{0} \times \frac{\pi}{180^{0}} = \frac{4\pi}{3}$
          \switchcolumn
          \item[c.] $210^{0} = 210^{0} \times \frac{\pi}{180^{0}} = \frac{7\pi}{6}$
          \item[f.] $-47^{0}30^{'} = -47\frac{1}{2}^{0} \\= \left(\frac{-95}{2}^{0}\right) \times \frac{\pi}{180^{0}} \\= \frac{-19\pi}{72} $
          \item[i.] $-7^{0}30^{'} = 7\frac{1}{2}^{0} \\= 7\frac{1}{2}^{0} \times \left(\frac{\pi}{180}\right)^{c} \\= \frac{15}{2} \times \left(\frac{\pi}{180}\right)^{c} \\= \left(\frac{\pi}{24}\right)^{c}$
          \switchcolumn
        \end{paracol}
\end{enumerate}

% example 3
\subsection{Sine, Cosine and Tangents:}
\paragraph{Without using calculator, solve the following questions:}
\begin{enumerate}
  \item[a.] If $sin(18^{0})= 0.309$, give another angle whose $sin$ is $0.309$.
        
        Since
        \[
        \begin{aligned}
          sin(180^0-\theta) = sin(\theta)
          &\therefore sin(180^{0}-18^{0}) = sin(162^{0}) \\
          &\therefore \theta = 162^{0}
        \end{aligned}
        \]

  \item[b.] If $cos(70^{0})= 0.342$, give another angle whose $cosine$ is $0.342$.

        Since
        \[
        \begin{aligned}
          cos\theta = cos(2\pi - \theta)
          &\therefore cos(360-70^{0}) = cos(290^{0}) \\
          &\therefore \theta = 290^{0}
        \end{aligned}
        \]

  \item[c.] If $cos(300^{0})= \frac{1}{2}$, give another angle whose $cosine$ is $\frac{1}{2}$.

        Since
        \[
        \begin{aligned}
          cos(300^{0}) &= cos(360-60^{0}) = cos(60^{0}) \\
                       &\therefore \theta = 60^{0}
        \end{aligned}
        \]
  \item[d.] If $cos(120^{0})= \frac{-1}{2}$, give another angle whose $cosine$ is $-0.5$.

        Since
        \[
        \begin{aligned}
          cos\theta = cos(360-\theta)
          &\therefore cos(120^{0}) = cos(360^{0} - 120^{0}) = cos(240^{0}) \\
          &\therefore \theta = 240^{0}
        \end{aligned}
        \]
  \item[e.] (Use Calculator) If $sin(x)= 0.848$, find two values for $x$ between $0^{0}$ and $360^{0}$.

        \[
        \begin{aligned}
          x = 58^{0}, 122^{0}
        \end{aligned}
        \]
\end{enumerate}

\end{document}
